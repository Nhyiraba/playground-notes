\documentclass[11pt,a4paper]{article}
\usepackage{polski}
\usepackage{indentfirst}
\usepackage{graphicx}
\usepackage{wrapfig}
\usepackage{listings}
\usepackage[cp1250]{inputenc}
\title{APIs: A Design Strategy (Jacobson, Brail, Woods)}
\author{Wojciech Gawronski (afronski@gmail.com)}

\begin{document}

\maketitle
\newpage

\tableofcontents
\newpage

\section{Introduction}

API is a way for two computer applications to talk to each other over a network (internet) using a common language that they both understand. API is a contract between them and this contract cannot be casually changed. API follows a specification - provider description, technical, business or additional constraints - API is a binding.

Types of API (definition for kind of API accessibility):
\begin{itemize}
	\item Private
	\item Public
\end{itemize}

\section{Business Strategy}

Why do you need an API?
\begin{itemize}
	\item You need a another client (desktop, mobile).
	\item Customers or partners ask for an API.
	\item Your site is getting screen scraped.
	\item You need more flexibility in providing content.
	\item You have data to make available.
	\item Competition has an API.
	\item You want to let potential partners test the waters.
	\item You want to scale integration with customers and partners.
	\item API improves the technical architecture.
\end{itemize}

\section{API Value Chain}

API should be an indirect channel for working with channel partners to reach end users.

Some questions should be answered before others:
\begin{itemize}
	\item Who is the API provider?
	\item How will the API be published and promoted?
	\item Who is the target audience for the API?
	\item What is their motiviation to consume the API?
	\item How will they benefit from it?
	\item What business assets are going to be provided through the API?
	\item What information, services, products will be available?
	\item Of What potential value could these assets be to others?
	\item How will the owner of the business assets benefits from the API?
	\item What types of apps will the API support?
	\item What features and functions will these apps have?
	\item Who will use the apps created using API?
	\item What benefit will they gain from using the apps?
	\item What benefits will the developers, end users and API provider get from their use?
\end{itemize}

In producing APIs should be involved whole chain of command (all departments with departements specialized in user interactions and Product Owner, Business Analysts).

\subsection{Creating a private API value chain}

API value chain elements:
\begin{itemize}
	\item Business Assets
	\item API provider
	\item Developers
	\item Applications
\end{itemize}

Usage:
\begin{itemize}
	\item Private API can be used to create apps to release to the public - many clients, many apps or highly customizable clients/dashboards/client proxies.
	\item Private APIs can be used by partners to create apps or to implement integration services. Many SaaS companies offer this type of API. Think of software companies that offer integration services with Salesforce - with this kind of APIs partners wasn't artificially bounded to the one domain.
	\item Private APIs can be used as a way to more efficiently build apps for internal use in an organization. For example, a large manufacturing company uses APIs to enable developers to build executive dashboards for distribution devices instead of requiring developers to request access to backend systems directly.
\end{itemize}

Benefits:
\begin{itemize}
	\item Private APIs can enable rapid and scalable development for mobile strategies, allowing each mobile product team to build apss quickly without worrying about how to populate them with content.
	\item Access to the business assets that are exposed in an API that leads to a quick return on any investment. Private APIs can simplyfy IT infrastructure to meet that demand.
	\item Private APIs can improve business development as they make it easier and faster partners to integrate.
\end{itemize}

Risks:
\begin{itemize}
	\item Risk for focusing only for self-serviceing.
	\item Lowering the productional support for "only" private API.
	\item Power of API is in people - when they use it you can improve it with directed and informative feedback.
	\item Education and evangelism for use of this API.
\end{itemize}

\subsection{Creating a public API value chain}

API value chain elements:
\begin{itemize}
	\item Business Assets - enhancening audience for business assets.
	\item API provider
	\item Developers
	\item Applications
\end{itemize}

Usage:
\begin{itemize}
	\item Enhancing value and extending the brand - API role in whole infrastructure is most important thing.
	\item Reaching niche markets.
	\item Expanding reach across platforms and devices.
	\item Fostering innovation.
\end{itemize}

Benefits:
\begin{itemize}
	\item Innovation and brand awareness.
	\item PR playground and new possibilities.
\end{itemize}

Risks:
\begin{itemize}
	\item Legal, technical, strategical risk.
	\item Rights infringement.
	\item Attacks against system and content (SPAM).
	\item Potential of cannibalization of core business.
	\item Overexposing your business assets to competitors.
	\item Conflicts in expectations with value proposition.	
	\item Resource allocation out of line with value proposition.
\end{itemize}

\subsection{Shifting between private and public}



\end{document}